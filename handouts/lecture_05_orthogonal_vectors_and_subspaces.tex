\documentclass{tufte-handout}
\usepackage{amsmath, amsfonts, amsthm}

\theoremstyle{definition} \newtheorem{definition}{Definition}
\newtheorem{theorem}{Theorem}
\theoremstyle{remark} \newtheorem{remark}{Remark}

\author{Gilbert Strang}
\title{Orthogonal vectors and subspaces}

\begin{document}

\maketitle

\section{Orthoganality of the Four Subspaces}
\begin{definition}[Orthogonal subspaces]
  Subspace $S$ is said to be \emph{orthogonal} to a subspace $T$, if
  every vector in $S$ is orthogonal to every vector in $T$.
\end{definition}

\begin{remark}
  Every vector $x$ in the \emph{nullspace} of $A$ is perpendicular to
  every row of $A$. The \emph{nullspace} $\mathrm{N}(A)$ and the
  \emph{row space} $\mathrm{C}(A^T)$ are orthogonal subspaces of $R^n$.
\end{remark}

\begin{remark}
  Every vector $y$ in the \emph{nullspace} of $A^T$ is perpendicular to
  every column of $A$. The \emph{nullspace} $\mathrm{N}(A^T)$ and the
  \emph{column space} $\mathrm{C}(A)$ are orthogonal subspaces of $R^m$.
\end{remark}

\section{Orthogonal Complements}
\begin{definition}[Orthogonal complement]
  The \emph{orthogonal complement} of a subspace $V$ contains every vector
  that is perpendicular to $V$. This orthogonal subspace is denoted by
  $V^{\perp}$.
\end{definition}


\begin{theorem}[Fundamental Theorem of Linear Algebra]
  $\mathrm{N}(A)$ is the orthogonal complement of the row space
  $\mathrm{C}(A^T)$(in $R^n$). $\mathrm{N}(A^T)$ is the orthogonal
  complement of the column space $\mathrm{C}(A)$(in $R^m$).
\end{theorem}

\end{document}